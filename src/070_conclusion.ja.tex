\chapter{結論}
\label{chap:conclusion}
本章では、本論文のまとめと今後の課題について述べる。

\section{本研究のまとめ}
\label{section:conclusion}
本研究では、保存時のWebブラウザ上の閲覧状態が失われてしまうという既存のブックマークシステムの問題を解決するため、Webブラウザ上のスクロール位置や、動画・音声・PDFをどこまで見ていたか、といったような覧状態を保存・復元できるブックマークシステムWeb Snapshotを提案・開発した。
Web SnapshotシステムはクライアントとしてiOSアプリケーションおよびChrome拡張機能を持ち、バックエンドにWebAPIを持つアプリケーションになっている。

実装したWeb Snapshotシステムを用いて、実際にWebページ・動画サイト・音声配信サービス・PDFをブックマークし、再開時に保存時の閲覧状態が復元するかどうか評価した。
その結果、スクロール位置は平均して約53\%の確率で復元に成功した。
動画の再生位置は平均して約47\%の確率で復元に成功した。
音声の再生位置は平均して約21\%の確率で復元に成功した。
PDFのページ数は平均して約70\%の確率で復元に成功した。

以上の結果から、本システムを利用することで、一定の条件下であればスクロール位置・動画の再生位置・PDFのページ数といったようなWebコンテンツの閲覧状態を復元できることがわかった。
一方で、デバイス間を跨いだ復元ではスクロール位置・動画・音声ともに復元に失敗するケースが多く、本システムの課題であることがわかった。
また、音声の再生位置の復元は多くのサービスで失敗しており、改善が必要である。

\section{今後の課題と展望}
本節では、本研究で提案したWeb Snapshotシステムの今後の課題と展望を述べる。

\subsection{スクロール位置の復元}
% デバイスを跨いだ復元ができない
% 広告への対応
% ユーザの利便性の問題
% ズーム率を復元できない
% ウィンドウサイズを変えるのではなく、cssで本文表示領域の幅を制御する

\subsection{スクロール位置の復元}
% iOSアプリケーション内ブラウザでの復元がうまくいかない
% 認証を必要とするページの復元が正しくできない。

\subsection{スクロール位置の復元}
% audioタグを使っていない音声配信サービスへの対応→調査
% iOSアプリケーション内ブラウザでの復元がうまくいかない
% 認証を必要とするページの復元が正しくできない。

\subsection{PDFのページ数の復元}
% Chrome拡張機能でしか使えない
% 絵や図などがあるPDF→切り取る処理を入れる
% 非ビジネス向けのPDF→独自にOCRの学習済みモデルを構築する
