\chapter{実装}
\label{chap:implementation}
本章では、Web-Snapshotシステムの実装について述べる。
はじめに実装環境について述べ、ついでクライアント側実装、サーバ側実装について説明する。

\section{実装環境}
本節では、本システムの実装環境について説明する。

クライアントの実装は、iOSアプリケーションはSwift、Chrome拡張機能はTypeScript/HTML/CSSを用いて実装した。

サーバ側は、WebAPIの実装にはRuby on Railsを使用し、Heroku\cite{}というPaaSにデプロイして運用している。データベースにはPostgreSQLを採用している。
PDFからページ数を抽出する機能については、Go言語で記述されたAWS LambdaからTextract\cite{}のAPIを呼び出している。

表\ref{tb:implementation-env}は実装環境の詳細である。

\begin{table}[htbp]
  \label{tb:implementation-env}
  \caption{実装環境}
  \begin{center}
    \begin{tabular}{|l|l|l|}
    \hline
    名目 & 使用技術 & バージョン \\ \hline
    実装言語 & Swift & n \\ \hline
    IDE & Xcode & n \\ \hline
    実装言語 & TypeScript & n \\ \hline
    実装言語 & HTML & n \\ \hline
    実装言語 & SCSS & n \\ \hline
    実装言語 & Ruby & n \\ \hline
    フレームワーク & Ruby on Rails & n \\ \hline
    RDBMS & PostgreSQL & n \\ \hline
    PaaS & Heroku & n \\ \hline
    実装言語 & Golang & n \\ \hline
    クラウドコンピューティングサービス & AWS Lambda & n \\ \hline
    ストレージ & AWS S3 & n \\ \hline
    OCR API & AWS Textract & n \\ \hline
    \end{tabular}
  \end{center}
\end{table}

\section{クライアント側実装}
クライアントはiOSアプリケーションおよびChrome拡張機能であり、それぞれ主にSwift、TypeScriptによって実装した。
それぞれのアプリケーションについて、ブックマーク保存モジュール・ブックマーク一覧モジュール・ブックマークの閲覧状態を復元するモジュールの3つのモジュールについて説明する。

\subsection{iOSアプリケーション実装}
本節では、iOSアプリケーションの実装について述べる。図\ref{}にiOSアプリケーションの実装の全体像を示す。

\subsubsection{認証モジュール}
本項では、認証モジュールについて説明する。あ。

% ユーザの利用の流れと画面。会員登録
ユーザが本アプリケーションを開くと、図\ref{}のような画面が表示される。この画面の”登録”ボタンを押すと図\ref{}が開き、会員登録できる。
会員登録のためには、メールアドレスとパスワードを入力する方法と、外部サービスを通じて登録する方法の2種類から選ぶことができる。

% メールアドレス 
メールアドレス。

% 外部サービス

% リクエスト。
%  レスポンス、認証情報の保存方法とその後のリクエストで使うこと
% ログイン

% ログイン
% 登録せずとも使用できること


\subsubsection{ブックマーク保存モジュール}
本項ではブックマーク保存モジュールについて説明する。
ブックマーク保存モジュールはユーザがブックマークしたいWebページのデータと閲覧状態を取得し、サーバに保存するモジュールである。
iOSアプリケーションでは、ブックマークの保存は標準のブラウザであるSafariから行う。
ユーザがブックマークしたいWebページを開いた状態で図\ref{}のような共有ボタンを押すと、共有機能に対応したアプリケーションのアイコンが並んで表示される。
この中から本アプリケーションのアイコンをクリックすると、図\ref{}のような確認用ポップアップが表示される。
ここで”保存”ボタンを押すと、Webコンテンツのデータの取得およびサーバへの保存が実行される。

上記の機能の実装にはAppleが提供しているShareExtension\cite{}を利用している。
ShareExtentionとは、Appleが提供するAppExtension\cite{app-extension}の1つで、任意のアプリケーション内のデータを開発者が提供するアプリケーションに共有できるAPIである。
ユーザが"保存"ボタンを押したタイミングで、ブラウザ上でJavaScriptコードを実行し、Webページのデータと閲覧状態を表すデータを取得する。
表\ref{tb:ios-data-js-api}に、JavaScriptで取得するデータと取得するために用いるAPIをまとめる。
なお、動画の再生位置はWebページ内のDOMにおいて一番初めに位置する動画のデータを取得する。Webページ内に動画が存在しない場合、ゼロ値は0である。

\begin{table}[htbp]
  \label{tb:ios-data-js-api}
  \caption{ブックマーク保存モジュールで取得するデータ}
  \begin{center}
    \begin{tabular}{|l|l|l|}
    \hline
    取得するデータ & 型 & API \\ \hline
    URL & String & document.URL \\ \hline
    ページタイトル & String & ca \\ \hline
    スクロール位置(縦) & Int & document.documentElement.scrollTop \\ \hline
    スクロール位置(横) & Int & document.documentElement.scrollLeft \\ \hline
    ページ長(縦) & Int & document.documentElement.scrollHeight \\ \hline
    ページ長(横) & Int & document.documentElement.scrollWidth \\ \hline
    動画再生位置 & Int & document.getElementsByTagName('video')[0].currentTime \\ \hline
    サムネイル画像URL & String & ca \\ \hline
    ズーム率 & Int & ca \\ \hline
    \end{tabular}
  \end{center}
\end{table}


ShareExtensionを通じて取得したデータは、iOSアプリケーション内のShareViewControllerというコントローラーに渡される。
ShareViewControllerは、取得したデータにいくつかのメタデータを付け加え、サーバに送信する。
ShareViewControllerが付与するメタデータは表\ref{tb:ios-meta-data}の通りである。

\begin{table}[htbp]
  \label{tb:ios-meta-data}
  \caption{ShareViewControllerが付与するメタデータ}
  \begin{center}
    \begin{tabular}{|l|l|}
    \hline
    サーバに送る値 & 型 \\ \hline
    ブラウザの種類 & String \\ \hline
    \end{tabular}
  \end{center}
\end{table}

図\ref{}にiOSアプリケーションにおけるブックマーク保存モジュールの全体像を示す。

\subsubsection{一覧モジュール}
本項ではブックマーク一覧モジュールについて説明する。
一覧モジュールでは、前項のブックマーク保存モジュールで保存したブックマークを一覧できる。
ユーザが本アプリケーションを開くと、サーバからユーザの保存したブックマークを取得する。サーバからのレスポンスとデータの概要を表\ref{tb:ios-bookmark-response}に示す。
取得したデータからタイトル・URL・保存日時・サムネイル画像を抽出し、図\ref{}のようにカード型で一覧表示する。

\begin{table}[htbp]
  \label{tb:ios-bookmark-response}
  \caption{サーバから取得するブックマークのデータ}
  \begin{center}
    \begin{tabular}{|l|l|}
    \hline
    取得する値 & 概要 \\ \hline
    id & 識別子 \\ \hline
    タイトル & Webページのタイトル \\ \hline
    タイトル & Webページのタイトル \\ \hline
    タイトル & Webページのタイトル \\ \hline
    タイトル & Webページのタイトル \\ \hline
    タイトル & Webページのタイトル \\ \hline
    タイトル & Webページのタイトル \\ \hline
    タイトル & Webページのタイトル \\ \hline
    タイトル & Webページのタイトル \\ \hline
    タイトル & Webページのタイトル \\ \hline
    タイトル & Webページのタイトル \\ \hline
    タイトル & Webページのタイトル \\ \hline
    タイトル & Webページのタイトル。 \\ \hline
    \end{tabular}
  \end{center}
\end{table}

ユーザがブックマークをクリックすると、閲覧状態復元モジュールに遷移し、ブックマークしたWebページを開いた上で閲覧状態を復元する。
ブックマークの削除やお気に入り登録、フォルダ分けなど、ブックマークアプリケーションで一般的に存在する機能も本モジュールで提供している。

\subsubsection{閲覧状態復元モジュール}
本項では閲覧状態復元モジュールについて説明する。
前項のブックマーク一覧モジュールで、ユーザが再度閲覧したいブックマークをクリックすると、閲覧状態復元モジュールが開く。

閲覧状態復元モジュールの機能の実装はWebViewControllerというコントローラーが担う。
WebViewControllerは、WKWebViewというiOSアプリ内専用のブラウザでブックマークしたWebページを表示する。
ブックマークが選択されると、WebViewControllerは保存時のスクロール位置/ページ長/ブラウザ/ズーム率および、WKWebView内のページ長/ズーム率などの情報から適切なスクロール位置を計算する。
この計算結果に基づいてWebページをスクロールすると、ブックマーク時点で画面上限に位置していたコンテンツが、WKWebViewの画面の上限にちょうど位置するように調整している。
このスクロール位置を計算するアルゴリズムは以下の通りである。

\begin{equation}
スクロール位置 = スクロール位置
\end{equation}

Webページの読み込みが完了すると、WebViewController内のwebView(\_:didFinish:)\cite{didFinish}というハンドラが呼び出される。
webView(\_:didFinish:)ハンドラは、閲覧状態の復元を司る関数で、スクロール位置と動画再生位置を設定する。
まず、webView(\_:didFinish:)ハンドラは、window.scrollToというAPIを利用して計算結果の分だけWebページをスクロールする。
その上で、Webページ内に動画が存在する場合は、videoタグのcurrentTimeに保存時の動画再生位置を設定する。
なお、保存時と同様、再生位置を復元するのはWebページ内のDOMの最上位の動画のみである。

\subsection{Chrome拡張機能実装}
本節では、Chrome拡張機能の実装について述べる。図\ref{}にChrome拡張機能の実装の全体像を示す。

\subsubsection{認証モジュール}

\subsubsection{ブックマーク保存モジュール}

\paragraph{PDFページ数保存機能}

\paragraph{テキストフラグメントへのリンク作成・保存機能}

\subsubsection{一覧モジュール}


\subsubsection{閲覧状態復元モジュール}

\paragraph{スクロール位置復元機能}


\paragraph{動画再生位置復元機能}


\paragraph{PDFページ数復元機能}

\section{サーバ側実装}


\subsection{WebAPI}
\begin{itemize}
  \item WebAPIの実装について説明する。
\end{itemize}

\subsubsection{認証API}
\subsubsection{ブックマークAPI}

% コンテンツの種類を判定
% 保存日の付与
% 共有用URLを生成

\subsubsection{フォルダAPI}

\subsection{PDFからページ数を抽出する機能}
本項では、PDFからページ数を抽出し、ブックマークとともに保存する機能について説明する。


% 保存時にキャプチャを撮る→ストレージに保存→LambdaからtextractのAPIを呼び出し、キャプチャ内の文字を抽出する→抽出した文字からページ数を撮る→ブックマークに保存する。
% 図で流れを説明する。

\section{まとめ}
本章では、Web-Snapshotシステムの実装について述べた。
次章では、本システムを利用しいて実際にWebページをブックマークし、閲覧状態の復元を評価する。