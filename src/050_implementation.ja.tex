\chapter{実装}
\label{chap:implementation}
本章では、Web-Snapshotシステムの実装について述べる。
はじめに実装環境について述べ、ついでクライアント側実装、サーバ側実装について説明する。

\section{実装環境}
本節では、本システムの実装環境について説明する。

クライアントの実装は、iOSアプリケーションはSwift、Chrome拡張機能はTypeScript/HTML/CSSを用いて実装した。

サーバ側は、WebAPIの実装にはRuby on Railsを使用し、Heroku\cite{heroku}というPaaSにデプロイして運用している。データベースにはPostgreSQLを採用している。
PDFからページ数を抽出する機能については、Go言語で記述されたAWS Lambda\cite{lambda}からTextract\cite{textract}のAPIを呼び出している。

表\ref{tb:implementation-env}は実装環境の詳細である。

% textlint-disable
\begin{table}[htbp]
  \label{tb:implementation-env}
  \caption{実装環境}
  \begin{center}
    \begin{tabular}{|l|l|l|}
    \hline
    名目 & 使用技術 & バージョン \\ \hline
    実装言語 & Swift & 5.5.1 \\ \hline
    IDE & Xcode & 13.1 \\ \hline
    実装言語 & TypeScript & 4.3.5 \\ \hline
    実装言語 & Ruby & 2.5.5 \\ \hline
    フレームワーク & Ruby on Rails & 5.2.6 \\ \hline
    RDBMS & PostgreSQL & 11.13 \\ \hline
    PaaS & Heroku & --- \\ \hline
    実装言語 & Golang & 1.17.0 \\ \hline
    クラウドコンピューティングサービス & AWS Lambda & --- \\ \hline
    ストレージ & AWS S3 & --- \\ \hline
    OCR & AWS Textract & --- \\ \hline
    \end{tabular}
  \end{center}
\end{table}
% textlint-enable



\section{クライアント側実装}
クライアントはiOSアプリケーションおよびChrome拡張機能であり、それぞれ主にSwift、TypeScriptによって実装した。
それぞれのアプリケーションについて、認証モジュール・ブックマーク保存モジュール・ブックマーク一覧モジュール・ブックマークの閲覧状態を復元するモジュールの4つのモジュールについて説明する。

\subsection{iOSアプリケーション実装}
本節では、iOSアプリケーションの実装について述べる。
iOSアプリケーションでは、スクロール位置の復元及び動画・音声の復元の3種類の復元に対応している。
iOSアプリケーションでは、以下の2つの理由でPDFの復元には対応していない。
1つ目の理由は、iOSアプリケーションからブラウザのスクリーンショットを保存できないためである。
2つ目の理由は、iOSアプリケーション内のブラウザがURLフラグメントによるPDFのページ数の変更に対応していないためである。

\subsubsection{認証モジュール}
本項では、認証モジュールについて説明する。
ユーザが本アプリケーションを開くと、図\ref{fig:impl-ios-top-not-auth-view}のような画面が表示される。この画面の”登録”ボタンを押すと図\ref{fig:impl-ios-auth-view}が開き、会員登録できる。
会員登録のためには、メールアドレスとパスワードを入力する方法と、外部サービスを通じて登録する方法の2種類から選ぶことができる。

\begin{figure}[htbp]
  \begin{tabular}{cc}
    \begin{minipage}[t]{0.45\hsize}
      \label{fig:impl-ios-top-not-auth-view}
      \begin{center}
        \includegraphics[bb=0 0 585 1266,width=5cm]{img/050_implementation/ios/ios-top-not-auth-view.pdf}
      \end{center}
      \caption{【iOS】初期画面}
    \end{minipage} &

    \begin{minipage}[t]{0.45\hsize}
      \label{fig:impl-ios-auth-view}
      \begin{center}
        \includegraphics[bb=0 0 585 1266,width=5cm]{img/050_implementation/ios/ios-auth-view.pdf}
      \end{center}
      \caption{【iOS】認証画面}
    \end{minipage}
  \end{tabular}
\end{figure}

メールアドレスで登録するには、図\ref{fig:impl-ios-auth-view}でメールアドレスとパスワードを入力する。
ユーザが必要な情報を入力した上で”登録”ボタンを押すと、AuthenticationViewControllerがサーバにデータを送信する。
会員登録が完了すると、サーバは認証に用いるトークンをヘッダに含めてレスポンスを返す。
AuthenticationViewControllerでは、取得したアクセストークンをKeyChain\cite{keychain}に保存する。
ヘッダにこのアクセストークンを付与してリクエストを送信することで、サーバが提供するAPIを利用できるようになる。

外部サービスを通じてログインする場合は、図\ref{fig:impl-ios-auth-view}の下部にあるボタンをクリックする。
例として、”Twitterで続ける”ボタンを押すと、Twitter\cite{twitter}の認証画面が開く。
本画面でTwitterアカウントにログインし、本アプリケーションとの連携を許可すると、OAuthの仕組みを利用して本アプリケーションに登録できる。

一度会員登録すると、以後アプリケーションを利用する際は自動でログインされる。
なお、iOSアプリケーションは会員登録せずとも利用できる。その場合、ブックマークのデータは端末内のストレージに保存される。

\subsubsection{ブックマーク保存モジュール}
本項ではブックマーク保存モジュールについて説明する。
ブックマーク保存モジュールはユーザがブックマークしたいWebページのデータと閲覧状態を取得し、サーバに保存するモジュールである。
iOSアプリケーションでは、ブックマークの保存は標準のブラウザであるSafariから行う。
ユーザがブックマークしたいWebページを開いた状態で図\ref{fig:usage-ios-share}のような共有ボタンを押すと、共有機能に対応したアプリケーションのアイコンが並んで表示される。
この中から本アプリケーションのアイコンをクリックすると、図\ref{fig:usage-ios-popup}のような確認用ポップアップが表示される。
ここで”保存”ボタンを押すと、Webコンテンツのデータの取得およびサーバへの保存が実行される。

上記の機能の実装にはShareExtension\cite{share-extension}を利用している。
ShareExtentionとは、Appleが提供するAppExtension\cite{app-extension}の1つで、任意のアプリケーション内のデータを開発者が提供するアプリケーションに共有できる拡張機能である。
ShareExtensionはブラウザ上でJavaScriptプログラムを実行し、Webページのデータを取得する。
表\ref{tb:impl-ios-data-js-api}に、JavaScriptで取得するデータと取得するために用いるAPIをまとめる。
なお、動画の再生位置はWebページのDOM内で最上位に位置する動画のデータのみを取得する。
音声についても同様とする。

% textlint-disable
\begin{table}[htbp]
  \label{tb:impl-ios-data-js-api}
  \caption{ブックマーク保存モジュールで取得するデータ}
  \begin{center}
    \begin{tabular}{|l|l|l|}
    \hline
    \multicolumn{1}{|c|}{\textbf{取得するデータ}} & \multicolumn{1}{|c|}{\textbf{型}} & \multicolumn{1}{|c|}{\textbf{API}} \\\hline
    URL & String & Document.URL \\ \hline
    ページタイトル & String & Document.title \\ \hline
    画面左に表示されているコンテンツのX座標 & Int & Element.scrollLeft \\ \hline
    画面左に表示されているコンテンツのY座標 & Int & Element.scrollTop \\ \hline
    ブラウザウィンドウの外側の幅 & Int & Window.outerWidth \\ \hline
    ブラウザウィンドウの外側の高さ & Int & Window.outerHeight \\ \hline
    動画再生位置 & Integer & HTMLMediaElement.currentTime \\ \hline
    音声再生位置 & Integer & HTMLMediaElement.currentTime \\ \hline
    \end{tabular}
  \end{center}
\end{table}
% textlint-enable

ShareExtensionを通じて取得したデータは、iOSアプリケーション内のShareViewControllerというコントローラーに渡される。
ShareViewControllerは、取得したデータにブラウザの情報をメタデータとして付け加え、サーバに送信する。

図\ref{fig:impl-ios-bookmark-save-module}にiOSアプリケーションにおけるブックマーク保存モジュールの全体像を示す。

\begin{figure}[htbp]
  \label{fig:impl-ios-bookmark-save-module}
  \begin{center}
    \includegraphics[bb=0 0 351 271,width=10cm]{img/050_implementation/ios/impl-ios-bookmark-save-module.pdf}
  \end{center}
  \caption{【iOS】ブックマーク保存モジュール全体像}
\end{figure}

\subsubsection{一覧モジュール}
本項ではブックマーク一覧モジュールについて説明する。
一覧モジュールでは、前項のブックマーク保存モジュールで保存したWebページを一覧できる。
ユーザが本アプリケーションを開くと、サーバからユーザの保存したブックマークを取得する。
取得したデータからタイトル・URL・保存日時・サムネイル画像を抽出し、図\ref{fig:usage-ios-top}のようにカード型で一覧表示する。

ユーザがブックマークをクリックすると、閲覧状態復元モジュールに遷移する。
ブックマークの削除やお気に入り登録、フォルダ分け、検索機能など、ブックマークアプリケーションで一般的に存在する機能も本モジュールで提供している。

\subsubsection{閲覧状態復元モジュール}
\label{sec:impl-client-ios-restore-module}

本項では閲覧状態復元モジュールについて説明する。
前項の一覧モジュールで、ユーザが再度閲覧したいブックマークをクリックすると、閲覧状態復元モジュールが開く。

本モジュールの機能はWebViewControllerというコントローラーが提供する。
WebViewControllerは、WKWebViewというiOSアプリ内専用のブラウザでブックマークしたWebページを表示する。
Webページの読み込みが完了すると、WebViewController内のwebView(\_:didFinish:)\cite{did-finish}というハンドラ関数が呼び出される。
この関数は、まずwindow.scrollToというAPIを利用して、保存時に取得した座標までWebページをスクロールする。
その上で、Webページ内に動画が存在する場合は、videoタグのcurrentTimeに保存時の動画再生位置を設定する。
なお、保存時と同様、再生位置を復元するのはWebページ内のDOMの最上位の動画のみである。
音声についても同様の処理を行う。


\subsection{Chrome拡張機能実装}
本節では、Chrome拡張機能の実装について述べる。
各モジュールの役割はiOSアプリケーションのそれと基本的に同様であるため、差分についてのみ説明する。

\subsubsection{認証モジュール}
本項では認証モジュールについて説明する。
ユーザが本拡張機能のアイコンをクリックすると、図\ref{fig:impl-chrome-auth-view}のような認証用のウィンドウが開く。
iOSアプリケーションと同様に、Chrome拡張機能でも本ウィンドウからメールアドレスか外部サービスを通じて会員登録・ログインできる。

\begin{figure}[htbp]
  \caption{【Chrome拡張機能】認証ウィンドウ}
  \label{fig:impl-chrome-auth-view}
  \begin{center}
    \includegraphics[bb=0 0 404 582,width=10cm]{img/050_implementation/chrome/impl-chrome-auth-view.pdf}
  \end{center}
\end{figure}

\subsubsection{ブックマーク保存モジュール}
本項ではブックマーク保存モジュールについて説明する。
Chrome拡張機能では、iOSアプリケーションで対応している3種類の復元に加えて、PDFのページ数の復元にも対応している。

ユーザが拡張機能のアイコンをクリックすると、図\ref{fig:usage-chrome-popup}のようなポップアップが表示される。
ユーザがポップアップ内の”保存する”ボタンをクリックすると、ボタンに登録されていたイベントが発火する。

このイベントは、まずWebページ内から必要な情報を取得する。
具体的には、Chrome Extensionのchrome.tabs API等を利用してWebページのタイトルなどのメタデータを取得する。
加えて、HTMLのAPIを利用して閲覧状態を表すデータを取得する。
この時に取得する情報と使用するAPIはiOSアプリケーションのブックマーク保存モジュールと同じである。

上記のデータの取得が成功すると、Chrome拡張機能は取得したデータをサーバに送信する。

ユーザがPDFを保存すると、本アプリケーションはURL等の基本情報を取得した上で、ブラウザの画面のキャプチャを撮影し、サーバ側のストレージに保存する。
画面のキャプチャには、chrome.tabs.captureVisibleTabというAPIを利用し、Chromeタブの表示領域の画像をエンコードしたデータのURLを取得する。
上記で取得したデータをpng画像に変換した上で、ストレージに送信する。
サーバ側では、この画面キャプチャからOCRを用いてページ数を抽出し、非同期的にデータベースに保存する。
そして、取得したページ数をURLのフラグメントとして付与する。
なお、ユーザが保存を試みたWebページがPDFであることの判定には、URLの拡張子を用いている。

なお、保存した画面キャプチャはページ数を取得した後、自動的に削除される。

\subsubsection{一覧モジュール}
本項ではブックマーク一覧モジュールについて説明する。
Chrome拡張機能では、保存したブックマークを確認できる画面を2種類用意している。

1つ目は、図\ref{fig:usage-chrome-popup}に示した、拡張機能のアイコンをクリックすると表示されるポップアップである。
ポップアップでは、保存したブックマークのうち最新の5つを表示する。

2つ目は、図\ref{fig:usage-chrome-list}に示した、ブックマークを一覧できる画面である。
上記のポップアップで”保存済みのコンテンツ一覧”というリンクを押すと、新しいタブが開き、この画面が表示される。
本画面では、iOSアプリケーションと同様にブックマークの削除やお気に入り登録・フォルダ分類・検索などの機能を提供している。
本画面は、HTML/CSSに加えて、アニメーションのためにjQuery\cite{jquery}・Bootstrap\cite{bootstrap}・Popper.js\cite{popper}等のライブラリを利用して実装している。

\subsubsection{閲覧状態復元モジュール}
本項では閲覧状態復元モジュールについて説明する。
前項の一覧モジュールで、ユーザが再度閲覧したいブックマークをクリックすると、それぞれのブックマークに登録されていたイベントが発火する。
このイベントは、まずChromeのtabsAPIのcreateメソッドを呼びだして新しいタブを開く。そして、そのタブでユーザが選択したWebページを表示する。
加えて、本イベントはChromeのruntimeAPI\cite{chrome-runtime-api}を利用してバックグラウンドスクリプト\footnote{ブラウザに関するイベントを監視し、処理を行うためのイベントベースのプログラム}にメッセージを送信する。
このメッセージを受信すると、本アプリケーションはブラウザウィンドウを保存時の大きさに設定する。
その上で、保存時に取得した座標まで画面をスクロールすることで、復元を実現する。
また、動画や音声が存在する場合は再生位置を復元する。
スクロール位置と動画/音声の再生位置の設定に用いるAPIも、iOSアプリケーションで用いるものと同様である。

なお、PDFについては、サーバ側で設定したURLを開くだけでブラウザ側が自動的にページ数を復元する。

\section{サーバ側実装}
サーバ側はWebAPIと、PDFからページ数を抽出するためのモジュールであり、それぞれRuby on Rails、Golangを用いて実装した。
WebAPIについては、認証関連API・コンテンツ関連API・フォルダ関連APIの3種類のAPIの実装について説明する。
その後、PDFからページ数を抽出するためのモジュールの実装について述べる。

\subsection{WebAPI}
本節では、WebAPIの実装について述べる。

\subsubsection{認証関連API}
本項では、認証に関連するAPIについて説明する。
メールアドレスおよびパスワードでの認証機能の実装には、Devise\cite{devise}、DeviseTokenAuth\cite{devise-token-auth}というgem\footnote{プログラミング言語Rubyにおけるライブラリのこと}を使用している。
会員登録用のAPIでは、リクエストにメールアドレス・パスワード・確認用パスワードの3つを要求する。
メールアドレスがすでに使用されている場合や、パスワードと確認用パスワードの値が異なる場合は、エラーレスポンスを返す。
リクエストに問題がない場合、パスワードのハッシュ値を計算し、データベースに保存する。
そして、ヘッダに認証用トークンを含めてレスポンスを返す。
コンテンツ関連APIやフォルダ関連APIを利用する場合は、この認証トークンをヘッダに追加してリクエストを送ることで、自動的に認証する。

ログイン用のAPIでは、リクエストとしてメールアドレスとパスワードを受け取り、パスワードのハッシュ値を計算する。
その計算結果とデータベースに保存されたパスワードのハッシュ値を比較し、一致した場合は認証する。

外部サービスを通じた認証機能の実装には、Omniauth\cite{omniauth}・Omniauth Twitter\cite{omniauth-twitter}・Omniauth GitHub\cite{omniauth-github}・Omniauth Google Oauth2\cite{omniauth-google-oauth2}というgemを使用している。

\subsubsection{コンテンツ関連API}
本項では、ブックマークの操作に関連するAPIについて説明する。

コンテンツ関連APIでは、ブックマークしたコンテンツに対するCRUD\footnote{データリソースの操作に必要とされる機能(Create/Read/Update/Delete)}を実行できるAPIを提供している。

加えて、ブックマークの保存時にコンテンツの種類を自動的に判定し、メタデータとして保存する。
この時、コンテンツの種類がPDFであった場合には、URLのフラグメントにページ数を付与する。
その他、サーバ側ではサムネイル画像が存在しない場合にデフォルトの画像URLを設定したり、保存日時の追加などの処理も行う。

また、保存された動画が、クエリパラメータによる再生位置の指定に対応しているサイトのものであった場合、URLのクエリパラメータに再生位置を付与する(例: \url{https://youtu.be/some-video?t=500})。
クライアント側では、このURLを開くだけで動画再生位置を復元できる。
クエリパラメータによる再生位置の指定に対応しているサイトの一覧と、クエリパラメータの形式を表\ref{tb:impl-server-video-query-params}にまとめる。

% textlint-disable
\begin{table}[htbp]
  \label{tb:impl-server-video-query-params}
  \caption{クエリパラメータによる再生位置の指定に対応しているサイトの一覧}
  \begin{center}
    \begin{tabular}{|l|l|}
    \hline
    \multicolumn{1}{|c|}{\textbf{動画サイト}} & \multicolumn{1}{|c|}{\textbf{クエリパラメータの形式}} \\\hline
    YouTube & ?t=330 \\ \hline
    ニコニコ動画 & ?from=330 \\ \hline
    OPENREC.tv & ?t=50 \\ \hline
    Dailymotion & ?start=50 \\ \hline
    Pornhub & \&t=50 \\ \hline
    xhamster & ?t=50 \\ \hline
    Twitch & ?t=00h05m30s \\ \hline
    \end{tabular}
  \end{center}
\end{table}
% textlint-enable

\subsubsection{フォルダ関連API}
本項では、フォルダの操作に関連するAPIについて説明する。
フォルダ関連APIでは、フォルダに対するCRUDを実行できるAPIを提供している。
加えて、フォルダにブックマークを追加・削除したり、特定のフォルダ内のブックマークを取得するAPIを提供している。

\subsection{PDFからページ数を抽出するモジュール}
本項では、PDFからページ数を抽出し、ブックマークとともに保存する機能について説明する。
クライアント側のブックマーク保存モジュールからPDFの画面キャプチャが送信されると、サーバ側はCarrierWave\cite{carrier-wave}というgemを使用してストレージにキャプチャをアップロードする。
ストレージにはAWSのS3\cite{s3}を使用している。

S3に画面キャプチャがアップロードされると、1つ目のAWS Lambdaがトリガーされる。
このLambdaは、前処理として保存された画面キャプチャの色を反転させ、結果として得られた画像を別のストレージに保存する。
この処理を行う理由は、ChromeではPDFのページ数を表示するナビゲーションバーは図\ref{fig:pdf-viewer-nav-bar-page-num}のように背景が暗くなっている。
背景が暗い画像は後に述べるTextract\cite{textract}というOCRのサービスにおいて認識率が低くなるため、このような処理を追加している。

前処理を行った画像がストレージに保存されると、2つ目のAWS Lambdaがトリガーされる。
このLambdaは、AWSのTextractというサービスのAPIを呼び出し、保存された画像をOCRにかける。
TextractはOCRによって抽出した文字列をDetectDocumentTextOutput\cite{detect-document-text-output}という構造体に含めて返却する。

LambdaはTextractのレスポンスから、PDFのページ数を抽出する。
DetectDocumentTextOutputは画像内に存在する英数字の一覧をリスト形式で持っている。
そのリストから、以下の正規表現\ref{regix-pdf-page-num}と初めにマッチする文字列を抽出する(例: 4/41)。
この文字列の初めの数字(例: 上記の例の場合、4)を、保存時のPDFのページ数とする。
PDFビューワでは画面上部のナビゲーションバーに現在のページ数を表示している。
そのため、正規表現にマッチした文字列のうち一番初めに現れたものが、PDFの現在のページ数である可能性が高い。

% textlint-disable
\begin{itembox}[l]{保存時のPDFのページ数を取得するために用いる正規表現}
  \label{regix-pdf-page-num}
  \begin{verbatim}
    // `3/10` or `3/ 10` or `3 /10` or `3 / 10`にマッチする
    `(\d+)( |)\/( |)\d+`
  \end{verbatim}
\end{itembox}
% textlint-enable

なお、ページ数の抽出が失敗した場合は、その時点で処理を中止し、アップロードされたファイルを削除する。

ページ数の抽出が完了すると、Lambdaは取得したPDFのページ数をWebAPIを通じてデータベースに保存する。
そのために、このLambdaはブックマークのpdf\_page\_numカラムのみを変更する権限を持つ。
保存が完了すると、Lambdaはアップロードされた画面キャプチャを削除する。

API側では、PDFのページ数が更新されると、そのページ数をURLのフラグメントとして付与する。
上記の一連の実装内容を図\ref{fig:impl-pdf-overall}にまとめる。

\begin{figure}[htbp]
  \label{fig:impl-pdf-overall}
  \begin{center}
    \includegraphics[bb=0 0 734 522,width=15cm]{img/050_implementation/server/impl-pdf-overall.pdf}
  \end{center}
  \caption{PDFページ数抽出機能の実装の全体像}
\end{figure}

