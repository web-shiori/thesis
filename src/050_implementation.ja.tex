\chapter{実装}
\label{chap:implementation}
本章では、Web-Snapshotシステムの実装について述べる。
はじめに実装環境について述べ、ついでクライアント側実装、サーバ側実装について説明する。

\section{実装環境}
本節では、本システムの実装環境について説明する。

クライアントの実装は、iOSアプリケーションはSwift、Chrome拡張機能はTypeScript/HTML/CSSを用いて実装した。

サーバ側は、WebAPIの実装にはRuby on Railsを使用し、Heroku\cite{}というPaaSにデプロイして運用している。データベースにはPostgreSQLを採用している。
PDFからページ数を抽出する機能については、Go言語で記述されたAWS LambdaからTextract\cite{}のAPIを呼び出している。

表\ref{tb:implementation-env}は実装環境の詳細である。

表には項目・使用技術・バージョンを書く
バックエンド(WebAPI): Ruby on Rails, Ruby, Heroku・postgreSQL

バックエンド(OCR): AWS(Lambda/CloudFormation/S3/Textract), Go

クライアント(iOSアプリ): Swift, XCode

クライアント(Chrome拡張機能): TypeScript, HTML, CSS

\section{クライアント側実装}
クライアントはiOSアプリケーションおよびChrome拡張機能であり、それぞれ主にSwift、TypeScriptによって実装した。
それぞれのアプリケーションについて、ブックマーク保存モジュール・ブックマーク一覧モジュール・ブックマークの閲覧状態を復元するモジュールの3つのモジュールについて説明する。

\subsection{iOSアプリケーション実装}
本節では、iOSアプリケーションの実装について述べる。

\subsubsection{認証モジュール}
本項では、認証モジュールについて説明する。

1. ユーザの利用の流れと画面。
2. リクエスト。

\subsubsection{ブックマーク保存モジュール}
本項ではブックマーク保存モジュールについて説明する。
\begin{itemize}
  \item ブックマーク保存モジュールの概要。
  \item ShareExtensionについて説明する。図と、保存の流れ、サーバにリクエストを送ること。
\end{itemize}

\paragraph{スクロール位置保存機能}
スクロール位置保存機能について説明する。
\begin{itemize}
  \item 保存時に、jsを実行して情報を取得することを説明する。取得する情報を表にする。
\end{itemize}

\paragraph{動画再生位置保存機能}
動画等の再生位置を保存する機能について説明する。
\begin{itemize}
  \item 保存時に、jsを実行して動画再生位置を取得することを説明する。取得する情報を表にする。
\end{itemize}

\subsubsection{ブックマーク一覧モジュール}
\begin{itemize}
  \item 保存時に、jsを実行して動画再生位置を取得することを説明する。取得する情報をリストアップ(表)する。
\end{itemize}

\paragraph{ブックマーク一覧機能}
\begin{itemize}
  \item 保存時に、jsを実行して動画再生位置を取得することを説明する。取得する情報をリストアップ(表)する。
\end{itemize}

\paragraph{フォルダ・お気に入り機能}
\begin{itemize}
  \item 保存時に、jsを実行して動画再生位置を取得することを説明する。取得する情報をリストアップ(表)する。
\end{itemize}

\subsubsection{ブックマーク閲覧状態復元モジュール}
\begin{itemize}
  \item 保存時に、jsを実行して動画再生位置を取得することを説明する。取得する情報をリストアップ(表)する。
\end{itemize}

\paragraph{スクロール位置復元機能}
\begin{itemize}
  \item 保存時に、jsを実行して動画再生位置を取得することを説明する。取得する情報をリストアップ(表)する。
\end{itemize}

\paragraph{動画再生位置復元機能}
\begin{itemize}
  \item 保存時に、jsを実行して動画再生位置を取得することを説明する。取得する情報をリストアップ(表)する。
\end{itemize}

\subsection{Chrome拡張機能実装}
本節では、Chrome拡張機能の実装について述べる。
\begin{itemize}
  \item 保存時に、jsを実行して動画再生位置を取得することを説明する。取得する情報をリストアップ(表)する。
\end{itemize}

\subsubsection{認証モジュール}

\subsubsection{ブックマーク保存モジュール}
\begin{itemize}
  \item 保存時に、jsを実行して動画再生位置を取得することを説明する。取得する情報をリストアップ(表)する。
\end{itemize}

\paragraph{スクロール位置保存機能}
\begin{itemize}
  \item 保存時に、jsを実行して動画再生位置を取得することを説明する。取得する情報をリストアップ(表)する。
\end{itemize}

\paragraph{動画再生位置保存機能}
\begin{itemize}
  \item 保存時に、jsを実行して動画再生位置を取得することを説明する。取得する情報をリストアップ(表)する。
\end{itemize}

\paragraph{PDFページ数保存機能}
\begin{itemize}
  \item 保存時に、jsを実行して動画再生位置を取得することを説明する。取得する情報をリストアップ(表)する。
\end{itemize}

\paragraph{テキストフラグメントへのリンク作成・保存機能}
\begin{itemize}
  \item 保存時に、jsを実行して動画再生位置を取得することを説明する。取得する情報をリストアップ(表)する。
\end{itemize}

\subsubsection{ブックマーク一覧モジュール}
\begin{itemize}
  \item 保存時に、jsを実行して動画再生位置を取得することを説明する。取得する情報をリストアップ(表)する。
\end{itemize}

\paragraph{ブックマーク一覧機能}
\begin{itemize}
  \item 保存時に、jsを実行して動画再生位置を取得することを説明する。取得する情報をリストアップ(表)する。
\end{itemize}

\paragraph{フォルダ・お気に入り機能}
\begin{itemize}
  \item 保存時に、jsを実行して動画再生位置を取得することを説明する。取得する情報をリストアップ(表)する。
\end{itemize}

\subsubsection{ブックマーク閲覧状態復元モジュール}
\begin{itemize}
  \item 保存時に、jsを実行して動画再生位置を取得することを説明する。取得する情報をリストアップ(表)する。
\end{itemize}

\paragraph{スクロール位置復元機能}
\begin{itemize}
  \item 保存時に、jsを実行して動画再生位置を取得することを説明する。取得する情報をリストアップ(表)する。
\end{itemize}

\paragraph{動画再生位置復元機能}
\begin{itemize}
  \item 保存時に、jsを実行して動画再生位置を取得することを説明する。取得する情報をリストアップ(表)する。
\end{itemize}

\paragraph{PDFページ数復元機能}
\begin{itemize}
  \item 保存時に、jsを実行して動画再生位置を取得することを説明する。取得する情報をリストアップ(表)する。
\end{itemize}

\section{サーバ側実装}
\begin{itemize}
  \item 保存時に、jsを実行して動画再生位置を取得することを説明する。取得する情報をリストアップ(表)する。
\end{itemize}

\subsection{WebAPI}
\begin{itemize}
  \item WebAPIの実装について説明する。
\end{itemize}

\subsubsection{認証API}
\subsubsection{ブックマークAPI}
\subsubsection{フォルダAPI}

\subsection{PDFからページ数を抽出する機能}
本項では、PDFからページ数を抽出し、ブックマークとともに保存する機能について説明する。

\begin{itemize}
  \item 保存時にキャプチャを撮る→ストレージに保存→LambdaからtextractのAPIを呼び出し、キャプチャ内の文字を抽出する→抽出した文字からページ数を撮る→ブックマークに保存する。
  \item 図で流れを説明する。
\end{itemize}

\section{まとめ}
本章では、Web-Snapshotシステムの実装について述べた。
次章では、本システムを利用しいて実際にWebページをブックマークし、閲覧状態の復元を評価する。