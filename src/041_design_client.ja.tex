\section{クライアント側設計}
本節では、クライアントを構成する4つのモジュールについて述べる。

1つ目は、認証モジュールである。
認証モジュールでは、ユーザの新規登録とログイン機能を提供する。
ユーザの認証情報はサーバに保存するため、ユーザは複数の端末から本アプリケーションを利用できる。

2つ目は、ブックマーク保存モジュールである。
ブックマーク保存モジュールでは、Webページをサーバに保存する。

ユーザが本モジュールを起動すると、Webコンテンツに対してJavaScriptプログラムを実行する。
プログラムは、WebページのタイトルやURLといった基本情報に加えて、スクロール位置や動画再生位置といったような閲覧状態に関する情報を取得する。
取得した情報はサーバのデータベースに保存される。
ブックマーク保存モジュールで取得するデータとその用途を表\ref{tb:design-save-bookmark-data-usage}に示す。

% textlint-disable
\begin{table}[htbp]
  \label{tb:design-save-bookmark-data-usage}
  \caption{ブックマーク保存モジュールで取得するデータとその用途}
  \begin{center}
    \begin{tabular}{|l|l|}
      \hline
      \multicolumn{1}{|c|}{\textbf{保存するデータ}} & \multicolumn{1}{|c|}{\textbf{用途}} \\\hline
      URL & ブックマーク機能実現 \\\hline
      ページタイトル & ブックマーク機能実現 \\\hline
      画面左に表示されているコンテンツのX座標 & 閲覧状態復元 \\\hline
      画面上に表示されているコンテンツのY座標 & 閲覧状態復元 \\\hline
      ブラウザウィンドウの外側の幅 & 閲覧状態復元 \\\hline
      ブラウザウィンドウの外側の高さ & 閲覧状態復元 \\\hline
      保存時の動画再生位置 & 閲覧状態復元 \\\hline
      保存時の音声再生位置 & 閲覧状態復元 \\\hline
    \end{tabular}
  \end{center}
\end{table}
% textlint-enable

3つ目は、ブックマーク一覧モジュールである。
一覧モジュールでは、前項のブックマーク保存モジュールで保存したWebページを一覧する画面を提供する。
表示されているブックマークをタップすると、ブックマーク閲覧状態復元モジュールに遷移する。
本モジュールでは、ブックマークの削除やお気に入り登録、任意のフォルダの作成およびコンテンツの整理が可能である。

4つ目は、閲覧状態復元モジュールである。
閲覧状態復元モジュールでは、ユーザが保存したブックマークをブラウザ上で開き、保存時の閲覧状態を復元する。
ブックマークが読み込まれたタイミングでJavaScriptプログラムを実行し、ブックマーク保存モジュールで取得したデータに基づいて閲覧状態を復元する。
スクロール位置の復元では、まずブラウザウィンドウを保存時の大きさに設定する。
その上で、保存時に取得した座標まで画面をスクロールすることで、復元を実現する。
動画や音声の再生位置についても、HTMLのAPIを利用して設定する。
PDFのページ数については、サーバ側でURLのフラグメントにページ数を付与しているため、そのURLをそのまま開くことで復元できる。
