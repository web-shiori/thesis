\subsection{動画の再生位置の復元の評価}
本項では、本研究で定義した『動画の閲覧状態の復元』の評価方法について説明する。

\subsubsection{評価に使用するWebページ}
まず、動画の再生位置の復元の評価に使用するWebページについて述べる。
動画の再生位置の復元の評価には、2021年のWebサイト別アクセス数ランキング\cite{The-50-Most-Visited-Websites-in-the-World}などの3つのサイト\cite{mmd-video-research}\cite{popular-video-service}を参考に、特に利用者が多く広く普及していると思われる動画サイト33個をピックアップし、使用する。
評価に用いる動画サービスの一覧を表\ref{tb:evl-video-service-list}に示す。
% なお、評価に使用した動画のURLは付録A\ref{chap:appendix-a}にすべて記載する。

% textlint-disable
\begin{table}[htbp]
  \label{tb:evl-video-service-list}
  \caption{評価に用いる動画サービス一覧}
  \begin{center}
    \begin{tabular}{|l|}
    \hline
    \multicolumn{1}{|c|}{\textbf{動画サービス}} \\\hline
    Amazon Prime Video \\ \hline
    Netflix \\ \hline
    hulu \\ \hline
    ABEMA \\ \hline
    Paravi \\ \hline
    TELASA \\ \hline
    FODプレミアム \\ \hline
    YouTube \\ \hline
    XVideos \\ \hline
    Pornhub \\ \hline
    xnxx \\ \hline
    Zoomレコーディング \\ \hline
    xhamster \\ \hline
    Twitch \\ \hline
    bilibili \\ \hline
    Tiktok \\ \hline
    ニコニコ動画 \\ \hline
    FC2動画 \\ \hline
    Dailymotion \\ \hline
    Gyao! \\ \hline
    ツイキャス \\ \hline
    パンドラTV \\ \hline
    PeeVee \\ \hline
    DMM \\ \hline
    rakutenTV \\ \hline
    TVer \\ \hline
    Mirrativ \\ \hline
    Mildom \\ \hline
    OPENREC.tv \\ \hline
    YOUKU \\ \hline
    Sina Video \\ \hline
    iQIYI \\ \hline
    Tencent Video \\ \hline
    \end{tabular}
  \end{center}
\end{table}
% textlint-enable

\subsubsection{評価方法}
次に、評価方法について説明する。
動画の再生位置の復元の評価では、実際にWebページ上の動画をブックマーク・再開し、動画の再生位置の復元の成功率を評価する。
『動画再生位置の復元』の定義は、第3章\ref{chap:web-snapshot-system-restore-definition}で説明した通りである。

なお、本実験の前提条件として、広告ブロッカーなどの拡張機能は無効の状態を仮定する。

\subsubsection{実験する条件}
動画の再生位置の復元の評価において、実験する条件を説明する。
本実験では、表\ref{tb:evl-video-audio-conditions}のように、Chrome拡張機能/iOSアプリケーションでの保存・復元を4パターン検証する。

% textlint-disable
\begin{table}[htbp]
  \label{tb:evl-video-audio-conditions}
  \caption{動画再生位置の復元の評価において実験する条件}
  \begin{center}
    \begin{tabular}{|l|}
    \hline
    \multicolumn{1}{|c|}{\textbf{実験条件}} \\\hline
    Chrome拡張機能で保存→Chrome拡張機能で復元 \\ \hline
    Chrome拡張機能で保存→iOSアプリケーションで復元 \\ \hline
    iOSアプリケーションで保存→Chrome拡張機能で復元 \\ \hline
    iOSアプリケーションで保存→iOSアプリケーションで復元 \\ \hline
    \end{tabular}
  \end{center}
\end{table}
% textlint-enable
