\begin{eabstract}

  Nowadays, it is commonplace for a single user to own and use multiple devices.
  Users may use a bookmarking system to view web content from multiple devices and browsers.
  However, the current bookmarking system can only save and restore web pages in URL units.
  Therefore, when a page is saved, the information about how far the user has browsed the contents is lost.
  As a result, users have to find the text or scene they are looking for every time they visit a web page.  

  Therefore, in this research, we propose a bookmarking system that can restore the browsing state at the time of saving when a user visits a bookmarked web page again.
  When a user bookmarks web content, information such as scrolling position, video playback position, and the number of PDF pages is saved along with the URL.
  When the user revisits the bookmarked web page, the user can restore the browsing state by setting the scroll position, video playback position, and the number of pages in the PDF file at the time of saving.

  In this study, we developed an iOS application and a Chrome extension as clients of the bookmarking system using Swift and TypeScript, respectively.
  In addition, we implemented a database server and a web API for storing bookmark data in Ruby.
  In addition, we implemented a module to extract the number of pages from a PDF using Golang.

  In order to evaluate the ability to restore the browsing state, we bookmarked various types of web pages using the implemented bookmarking system.
  Then, we reviewed the bookmarked web pages on multiple devices and browsers to evaluate the ability to restore them.

  As a result, the scrolling position was restored at about 86% when both saving and resuming were done with the Chrome extension.
Different browser window sizes were restored in the same way.
When both saving and resuming were done with the iOS application, the recovery rate was about 54%.
When saving was done with the iOS application and resuming with the Chrome extension, the recovery rate was about 7\%.
When saving was done with the Chrome extension and resuming was done with the iOS application, all sites could not be restored.

When both saving and resuming were done with the Chrome extension, the video playback position was restored at a rate of about 70 percent.
When both saving and resuming were done using the iOS application, the recovery rate was about 35%.
When saving was done with the iOS application and resuming with the Chrome extension, the recovery rate was about 62%.
When saving was done with the Chrome extension and resuming was done with the iOS application, the recovery rate was about 21%.

The audio playback position was restored at about 33% when both saving and resuming were done with the Chrome extension.
When both saving and resuming were done using the iOS application, the recovery rate was about 13%.
When saving was done with the iOS application and resuming with the Chrome extension, the recovery rate was about 26%.
When saving was done with the Chrome extension and resuming was done with the iOS application, the recovery rate was about 13%.

On average, the number of pages in a PDF was successfully restored about 70 percent of the time.
However, restoring the number of pages in a PDF could only be achieved with the Chrome extension.

  These results show that our system can recover the browsing status of web contents, such as scrolling position, video playback position, and the number of pages in a PDF, under certain conditions.
  On the other hand, there are many cases where the system fails to restore the scroll position, video, and audio when restoring across devices, which is a problem for our system.
  In addition, many services fail to restore the playback position of audio, which needs to be improved.

\end{eabstract}
