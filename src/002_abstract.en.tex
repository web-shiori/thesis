\begin{eabstract}

  (Google翻訳です。本綴じまでに書き直します。)

  Nowadays, it is common for one user to own and use multiple terminals.
  Users may use the bookmark system to browse Web content from multiple terminals / browsers.
  However, with the current bookmark system, Web pages can only be saved and restored in URL units.
  As a result, information about how far the user was browsing the content when the page was saved is lost.
  As a result, each time a user visits a Web page, he or she must re-find the desired wording or scene in the content.

  Therefore, in this research, I propose a bookmark system that can restore the browsing state at the time of saving when the user visits the bookmarked Web page again.
  When a user bookmarks Web content, information such as scroll position, video playback position, and number of PDF pages is saved along with the URL.
  When the user visits the bookmarked Web page again, the browsing state is restored by setting the scroll position, video playback position, number of PDF pages, etc. at the time of saving.

  In this research, I developed iOS application and Chrome extension with Swift and TypeScript as a client of bookmark system.
  In addition, I implemented a database server and Web API for storing bookmark data in Ruby.
  In addition, I implemented a module to extract the number of pages from PDF in Go language.

  In order to evaluate the ability to restore the browsing state, I bookmarked various types of Web pages using the implemented bookmark system.
  Then, I browsed the bookmarked Web page again on multiple terminals and browsers.

  (Write the details after the evaluation.)
  As a result, it turned out to be 〇〇.

  (Summary of conclusions.)

\end{eabstract}
