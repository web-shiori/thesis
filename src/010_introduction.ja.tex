\chapter{序論}
\label{chap:introduction}
本章では,はじめに本研究における背景を述べる.ついで,問題意識および本研究の目的を述べる.最後に本論文の構成を示す.

\section{背景}
\label{section:background}
Webブラウジングにおいて、様々な方法でWebページ内から目当てのコンテンツを指し示すことが可能である
\begin{itemize}
    \item アンカーリンク
    \item youtube等で再生位置をクエリパラメータとして指定して共有可能
    \item Googleがテキストフラグメントを指定して共有できるChrome拡張機能をリリースした
    \item PDFでも、フラグメントでページ数を指定することで特定のページを共有することが可能
\end{itemize}

\section{本研究が着目する課題}
既存のブックマークシステムは、ページ単位でしか保存・復元することができない
その結果、以下のような問題が発生する
\begin{itemize}
        \item 一度ページから離れると、閲覧していたWebサイトや動画・PDFなどをどこまで読んだか・視聴したかがわからなくなってしまう
        \item Webサイトや動画・PDFの特定の文言・シーンを共有したり、見つけ直すのに手間がかかる
        \item 複数の異なるブラウザ間でWebサイトや動画の閲覧の進捗を同期させることができない
\end{itemize}

\section{本研究の目的とアプローチ}
\begin{itemize}
      \item  Webページとともにスクロール位置・動画再生位置・PDFのページ数などの閲覧状態を保存・復元することができるブックマークシステムを開発する
\end{itemize}

\section{本論文の構成}

本論文における以降の構成は次の通りである。
\ref{chap:related_works}章では、

\ref{chap:web_snapshot_system}章では、

\ref{chap:design}章では、本研究で提案するシステムの設計について述べる。

\ref{chap:implementation}章では、

\ref{chap:evaluation}章では、

\ref{chap:conclusion}章では、本研究における結論を示す。