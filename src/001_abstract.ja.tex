\begin{jabstract}

  現在では一人のユーザが複数の端末を所有して使い分けることが当たり前になっている。
  ユーザはWebコンテンツを複数の端末・ブラウザから閲覧するために、ブックマークシステムを利用することがある。
  しかし、現在のブックマークシステムでは、WebページをURL単位でしか保存・復元できない。
  そのため、ページを保存した際にユーザがどこまでコンテンツを閲覧していたかという情報が失われてしまう。
  その結果、ユーザはWebページを訪れる度に、コンテンツ内の目当ての文言・シーンを見つけ直さなければならない。  

  そこで本研究では、ユーザがブックマークしたWebページを再度訪れた際に、保存時の閲覧状態を復元できるブックマークシステムを提案する。
  ユーザがWebコンテンツをブックマークすると、URLとともにスクロール位置・動画再生位置・PDFのページ数などの情報を保存する。
  ユーザがブックマークしたWebページを再度訪れた際に、保存時のスクロール位置・動画再生位置・PDFのページ数などを設定することで、閲覧状態を復元する。

  本研究では、ブックマークシステムのクライアントとしてiOSアプリケーション・Chrome拡張機能をそれぞれSwift・TypeScriptで開発した。
  加えて、ブックマークデータを保存するためのデータベースサーバとWebAPIをRubyで実装した。
  さらに、PDFからページ数を抽出するモジュールをGolangで実装した。

  閲覧状態を復元できることを評価するため、実装したブックマークシステムを用いて様々な種類のWebページをブックマークした。
  そして、Macの上のChromeとiOSのSafariでブックマークしたWebページを再度閲覧し、復元できることを評価した。

 その結果、スクロール位置は保存・再開をともにChrome拡張機能で行った場合、約86\%の復元率だった。
ブラウザのウィンドウサイズが異なる場合も、同様に復元できた。
保存・再開をともにiOSアプリケーションで行った場合、約54\%の復元率だった。
保存をiOSアプリケーション、再開をChrome拡張機能で行った場合、約7\%の復元率だった。
保存をChrome拡張機能、再開をiOSアプリケーションで行った場合は、全てのサイトで復元できなかった。

動画の再生位置は保存・再開をともにChrome拡張機能で行った場合、約70\%の復元率だった。
保存・再開をともにiOSアプリケーションで行った場合、約35\%の復元率だった。
保存をiOSアプリケーション、再開をChrome拡張機能で行った場合、約62\%の復元率だった。
保存をChrome拡張機能、再開をiOSアプリケーションで行った場合は、約21\%の復元率だった。

音声の再生位置は保存・再開をともにChrome拡張機能で行った場合、約33\%の復元率だった。
保存・再開をともにiOSアプリケーションで行った場合、約13\%の復元率だった。
保存をiOSアプリケーション、再開をChrome拡張機能で行った場合、約26\%の復元率だった。
保存をChrome拡張機能、再開をiOSアプリケーションで行った場合は、約13\%の復元率だった。

PDFのページ数は平均して約70\%の確率で復元に成功した。
ただし、PDFのページ数の復元はChrome拡張機能でのみ実現できた。

  以上の結果から、本システムを利用することで、一定の条件下であればスクロール位置・動画の再生位置・PDFのページ数といったようなWebコンテンツの閲覧状態を復元できることがわかった。
  一方で、端末間を跨いだ復元ではスクロール位置・動画・音声のいずれも復元に失敗するケースが多く、本システムの課題であることがわかった。
  また、音声の再生位置の復元は多くのサービスで失敗しており、改善が必要である。

\end{jabstract}
