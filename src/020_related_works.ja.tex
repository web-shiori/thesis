\chapter{関連技術}
\label{chap:related_works}

本章では、既存のブックマークシステムを整理し、本研究との差分について説明する。

\section{ブラウザの提供するブックマーク機能}
SafariやChromeなどの標準的なブラウザは、2種類のブックマーク機能を提供している。

1つ目は、ブックマーク機能である。
ブックマーク機能は、任意のWebページへのショートカットをブラウザ内に保存する機能である。
保存したブックマークはブラウザのトップ画面やナビゲーションバーなどに表示される。
本機能は頻繁に使用するWebページへ高速にアクセスする目的で使用される。
Safariにおいてブックマーク機能を利用している画面を図\ref{}に示す。

2つ目は、リーディングリスト機能である。
リーディングリスト機能は、後で読みたいWebページを保存しておくために使用される。
ブックマーク機能との違いは、リーディングリストに追加されるWebページは頻繁に使用されるわけではない点である。
ユーザはリーディングリストに追加したWebページを閲覧した後、リーディングリストから削除することが多い。

これらの機能では、単にWebページのURLを保存する。
そのため、保存時の閲覧状態は失われてしまう。
また、データをブラウザ内に保存するため、異なるブラウザ間では使用できない。

\section{ブックマーク機能を提供するWebサービス}
% ブックマークを保存するWebサービス。デバイスやブラウザに依存しない。どこからでも使えるのが特徴
% これは本システムと同様
% 閲覧状態は失われる。

ブラウザの提供するブックマーク機能では、異なるブラウザ間でブックマークを共有できないという欠点があった。
その課題を解決するために誕生したのが、Webサービスとしてのブックマーク機能である。
これらのWebサービスでは、ユーザが保存したブックマークをオンライン上で保存する。
それによって、ユーザはブラウザや端末をまたいでブックマークを利用できる。
ブックマーク機能を提供するWebサービスのうち、著名なものを紹介する。

\subsection{Pocket}
PocketはMozilla\cite{}が提供しているWebサービスである。
Pocketでは複数の端末やブラウザからブックマークを利用できる一方、保存時の閲覧状態は失われてしまう。
Pocketでブックマーク機能を利用している画面を図\ref{}に示す。

\subsection{Instapaper}
InstapaperはMarcoArment社\cite{}が提供しているWebサービスである。
このサービスの特徴は、ブックマークをオフラインで利用できる点である。
Instapaperでも、Pocket同様に保存時の閲覧状態は失われてしまう。
Instapaperでブックマーク機能を利用している画面を図\ref{}に示す。

\section{ソーシャルブックマーク}
上記に挙げた例以外にも、ソーシャルブックマークというブックマークシステムも存在する。
ソーシャルブックマークとは、インターネット上で自分のブックマークを公開し、他のユーザと共有するサービスである\cite{}。
ソーシャルブックマークの例として、はてな社\cite{}が提供するはてなブックマーク\cite{}というサービスがある。
インターネット上でブックマークを共有するというサービスの特性上、閲覧状態は記録できない。
はてなブックマークのサービス画面を図\ref{}に示す。

\section{まとめ}
本章では、既存のブックマークシステムについて整理し、本研究との差分を説明した。
次章では、本研究で提案するシステムについて述べる。