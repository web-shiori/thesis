
\subsection{Chrome拡張機能実装}
本節では、Chrome拡張機能の実装について述べる。
各モジュールの役割はiOSアプリケーションと基本的に同様であるため、差分についてのみ説明する。

\subsubsection{認証モジュール}
本項では認証モジュールについて説明する。
ユーザが本拡張機能をChromeにインストールした後、本アプリケーションのアイコンをクリックすると、図\ref{fig:impl-chrome-auth-view}のような認証用のウィンドウが開く。
iOSアプリケーションと同様に、Chrome拡張機能でも本ウィンドウからメールアドレスか外部サービスを通じて会員登録・ログインできる。

\begin{figure}[htbp]
  \caption{【Chrome拡張機能】認証ウィンドウ}
  \label{fig:impl-chrome-auth-view}
  \begin{center}
    \includegraphics[bb=0 0 404 582,width=10cm]{img/050_implementation/chrome/impl-chrome-auth-view.pdf}
  \end{center}
\end{figure}

\subsubsection{ブックマーク保存モジュール}
本項ではブックマーク保存モジュールについて説明する。
本アプリケーションのChrome拡張機能では、iOSアプリケーションで対応している2種類の復元に加えて、PDFのページ数の復元にも対応している。
本項では、Chrome拡張機能における基本的なブックマーク保存機能に加えて、PDFページ数の取得機能についても説明する。

ユーザが拡張機能のアイコンをクリックすると、図\ref{fig:usage-chrome-popup}のようなポップアップが表示される。
ユーザがブックマークしたいWebページを開いた状態でポップアップ内の”保存する”ボタンをクリックすると、ボタンに登録されていたイベントが発火する。

このイベントは、まずWebページ内から必要な情報を取得する。
具体的には、Chrome Extensionのchrome.tabs API等を利用してWebページのタイトルなどのメタデータを取得したり、HTMLのAPIを利用して閲覧状態を表すデータを取得する。
この時に取得する情報と使用するAPIはiOSアプリケーションのブックマーク保存モジュールと同じである。

上記のデータの取得が成功すると、Chrome拡張機能は取得したデータをサーバに送信する。

ユーザがWeb上にホスティングされているPDFを保存すると、本アプリケーションはURL等の基本情報を取得した上で、ブラウザの画面のキャプチャを撮影し、ストレージに保存する。
キャプチャには、というAPIを利用し、Chromeに表示されている画面を撮影する。
サーバ側では、ストレージにファイルがアップロードされたことをトリガーとして、画面キャプチャからOCRを用いてPDFのページ数を取得し、非同期的にデータベースに保存する。
なお、ユーザが保存を試みたWebページがPDFであることの判定には、URLの拡張子を用いている。

また、プライバシーの確保のため、画面キャプチャ時には図\ref{}のようなアラートを表示した上で、ユーザがオプトアウトできるようになっている。
保存した画面キャプチャはページ数を取得した後、自動的に削除される。

\subsubsection{一覧モジュール}
本項ではブックマーク一覧モジュールについて説明する。
Chrome拡張では、保存したブックマークを確認できる画面を2種類用意している。

1つ目は、図\ref{fig:usage-chrome-popup}に示した、拡張機能のアイコンをクリックすると表示されるポップアップである。
ポップアップでは、保存したブックマークのうち最新の5つを表示する。

2つ目は、図\ref{fig:usage-chrome-list}に示した、ブックマークをすべて一覧できる画面である。
上記のポップアップで”保存済みのコンテンツ一覧”というリンクを押すと、新しいタブが開き、この画面が表示される。
本画面では、iOSアプリケーションと同様にブックマークの削除やお気に入り登録・フォルダ分類・検索などの機能を提供している。
本画面は、HTML/CSSに加えて、アニメーションのためにjQuery\cite{jquery}・Bootstrap\cite{bootstrap}・Popper.js\cite{popper}等のライブラリを利用して実装している。

\subsubsection{閲覧状態復元モジュール}
本項では閲覧状態復元モジュールについて説明する。
前項のブックマーク一覧モジュールで、ユーザが再度閲覧したいブックマークをクリックすると、それぞれのブックマークに登録されていたイベントが発火する。
このイベントは、まずChromeのtabsAPIのcreateメソッドを呼びだして新しいタブを開く。そして、そのタブでユーザが選択したWebページを表示する。
加えて、本イベントはChromeのruntimeAPI\cite{chrome-runtime-api}を利用してバックグラウンドスクリプト\footnote{}にメッセージを送信する。
バックグラウンドスクリプトとは、ブラウザブラウザに関するイベントを監視し、処理を行うためのイベントベースのプログラムである。
このメッセージを受信すると、バックグラウンドページで適切なスクロール量を計算する。
この計算に用いるアルゴリズムはiOSアプリケーションの閲覧状態復元モジュールの項\ref{sec:impl-client-ios-restore-module}で説明したものと同様である。
本アプリケーションは計算結果に基づいてページをスクロールし、動画が存在する場合は再生位置を復旧する。
スクロール位置と動画再生位置の設定に用いるAPIも、iOSアプリケーションの閲覧状態復元モジュールの項\ref{sec:impl-client-ios-restore-module}で説明したものと同様である。

なお、PDFについては、サーバ側で設定したURLを開くだけでブラウザ側が自動的にページ数を復元する。
