\chapter{Web-Snapshot システム}
\label{chap:web_snapshot_system}
本章では、コンテンツの閲覧状態を保存できるブックマークシステム、Web-Snapshotを提案する。
はじめにWeb-Snapshotシステムの概要を述べ、次にWeb-Snapshotの特徴を説明する。
そして最後に、ユーザがWeb-Snapshotを利用する流れについて述べる。

\section{システムの概要}
Web-Snapshotはコンテンツの閲覧状態を保存できるブックマークアプリケーションである。
Webコンテンツをブックマークすると、URLとともにスクロール位置・動画再生位置・PDFのページ数などの情報を保存する。
保存されたブックマークはiOSアプリケーションおよびChrome拡張機能から閲覧・管理することができる。
ブックマークしたコンテンツを再度開くと、保存時のスクロール位置・動画再生位置・PDFのページ数を復元する。

\section{システムの特徴}
本節では、Web-Snapshotの特徴としてあげられる機能を3つ挙げる。

\subsection{スクロール位置保存・復旧機能}
Web-Snapshotは、ブックマーク時のコンテンツのスクロール位置を保存・復元する。
本アプリケーションでは、ブックマークの復元時にデバイスの画面サイズ・ズーム率・使用ブラウザ等の情報をもとにスクロール位置を再計算する。
そのため、異なるブラウザ間やデバイス間でもWebコンテンツの閲覧状態を正しく復元することができる。

\subsection{動画再生位置保存・復旧機能}
Web-Snapshotは、ブックマークしたWebコンテンツ内に含まれる動画の再生位置を保存・復元する。
ただし、動画再生位置を復元できるのは、WebコンテンツのDOM内で一番はじめに位置する動画についてのみである。

\subsection{PDFページ数保存・復旧機能}
Web-Snapshotは、ブックマークしたPDFのページ数を保存・復元する。

\subsection{テキストへのリンク作成・保存機能}
\subsection{共有機能}

\section{システムの使用方法}
本節では、Web-Snapshotシステムの使用方法を、iOSアプリケーション版、Chrome拡張機能版のそれぞれについて説明する。

\subsection{iOSアプリケーション}
本アプリケーションをインストールしたiOS端末において、Safariの「共有ボタン」(\ref{safari-share-button})を押すと、\ref{fig:}のように本アプリケーションのアイコンが表示される。
ブックマークしたいWebページを開いた状態でこのアイコンをクリックすると、\ref{fig:}のように確認用ポップアップが表示される。ここで"保存する"ボタンを押すことで、Webページをブックマークすることができる。

本アプリケーションを開くと、\ref{usage-ios-top}のようなトップ画面が開かれる。ユーザはここで保存したブックマークを一覧することができる。
その他にも、ブックマークをお気に入りに登録したり、フォルダを作成して整理することも可能である。

ブックマークを再度閲覧するには、閲覧したいブックマークをクリックする。すると、アプリケーション内のブラウザに遷移し、保存したWebページを閲覧することが可能になる。
Webページがロードされると、スクロール位置・動画再生位置・PDFのページ数などの情報をもとにブックマーク時の閲覧状態を復元する。
閲覧画面を\ref{fig:}に示す。

\subsection{Chrome拡張機能}
本アプリケーションをChromeにインストールすると、\ref{fig:}のように本アプリケーションのアイコンが表示される。

アイコンをクリックすると、\ref{fig:}のようなポップアップが表示される。ポップアップには、"保存する"ボタンと"保存済みのコンテンツ一覧"という2つのボタンに加えて、保存済みのコンテンツが5つまで表示される。
"保存する"ボタン

\section{まとめ}
本章では、コンテンツの閲覧状態を保存できるブックマークシステム、Web-Snapshotを提案した。
また、Web-Snapshotシステムの特徴および使用方法を述べた。次章では、本システムの設計について述べる。
