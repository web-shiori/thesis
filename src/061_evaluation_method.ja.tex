\section{評価手法}
本研究では、Webページの閲覧状態の復元の定義を表\ref{tb:re-intro-restore-definition}のように定義した。

% textlint-disable
\begin{table}[htbp]
  \begin{center}
    \caption{【再掲】本研究における閲覧状態の復元の定義}
    \label{tb:re-intro-restore-definition}
    \begin{tabular}{|c|l|l|l|}
      \hline
      復元する項目 & 条件 &  復元の定義 \\\hline\hline
      スクロール位置 & 全て & 
        \begin{tabular}{l}
          保存時の画面左上に位置しているコンテンツが\\画面左上に表示されている。 
        \end{tabular}\\\hline
      動画再生位置 & 動画 & 保存時の再生位置(秒単位)から動画がはじまる。 \\\hline
      音声再生位置 & 音声 & 保存時の再生位置(秒単位)から音声がはじまる。 \\\hline
      ページ数 & PDF & 保存時のページが開く。 \\\hline
    \end{tabular}
  \end{center}
\end{table}
% textlint-enable

今回の評価実験では、実際にWebページをブックマーク・再開し、本研究における復元の定義を満たしているか目視で確認する。
まず、スクロール位置の復元について様々な種類のWebページを用いて実験する。
複数端末間・Chrome/Safari間での復元など、様々なシチュエーションでの復元についても評価する。

動画/音声については、それぞれ利用者数の多いWebサービスについて実験する。
動画については、利用者数の多いYouTube\cite{youtube}/Amazon Prime Video\cite{amazon-prime-video}/Netflix\cite{netflix}/ABEMA\cite{abema}で再生位置が復元することを評価する。

音声については、インターネットラジオのradiko\cite{radiko}。
PodcastのホスティングサービスのApple Podcast\cite{apple-podcast}。
音楽ストリーミングサービスのSpotify\cite{spotify}。
オーディオブックのAudible\cite{audible}で音声の再生位置が復元するか検証する。

PDFについてはOCRを利用しており不確実であるため、いくつかの種類のPDFを用いて実験した上で、正しくページ数を復元できる確率を示す。

\subsection{スクロール位置の復元}
本項では、本研究で定義したスクロール位置の復元の評価方法について説明する。

% 評価項目・手法
まず、評価手法について説明する。
スクロール位置の復元の評価では、縦横のスクロール位置がそれぞれ復元することを目視で確認する。
なお、『スクロール位置の復元』の定義は、第3章\ref{chap:web-snapshot-system-restore-definition}で説明した通りである。

% 実験条件
本評価では、様々な条件下で正しくスクロール位置が復元することを実験する。
具体的には、まず同じ端末上で保存時と再開時のウィンドウサイズ・画面の向き・ブラウザを揃えて実験する。
Chrome拡張機能版とiOSアプリケーション版のそれぞれについて評価する。

次に、ウィンドウサイズを変えて実験する。
ウィンドウサイズの縦横比が保存時と再開時で異なる場合でも、スクロール位置が正しく復元するか検証する。

さらに、画面の向きを変えてテストする。
現在のモバイル端末では、画面の向きを変更できる。
保存時と再開時の端末の向きが異なる場合でも、スクロール位置を正しく復元できるか評価する。

異なるブラウザ間での復元について評価する。
本システムはPC上のChromeおよびiOS端末上のSafariのみに対応している。
そのため、使用するブラウザはChrome/Safariのみである。

異なる端末間での復元について検証する。
実験ではiPhone12\cite{iphone12}およびiPad(10.2インチ)\cite{ipad}
なお、PCとモバイル端末間での比較は、Chrome/Safari間の実験で実施する。

本実験で用いる条件と実験パターンを表\ref{tb:evl-scroll-position}にまとめる。

% textlint-disable
\begin{table}[htbp]
  \label{tb:evl-scroll-position}
  \caption{実験する条件}
  \begin{center}
    \begin{tabular}{|l|l|}
    \hline
    テスト条件  \\ \hline
    再開時のウィンドウサイズが保存時より縦に長い  \\ \hline
    再開時のウィンドウサイズが保存時より縦に短い  \\ \hline
    再開時のウィンドウサイズが保存時より横に長い  \\ \hline
    再開時のウィンドウサイズが保存時より横に短い  \\ \hline
    再開時のウィンドウサイズが保存時より縦横に長い  \\ \hline
    再開時のウィンドウサイズが保存時より縦横に短い  \\ \hline
    保存時は画面の向きが縦/再開時は画面の向きが横 \\ \hline
    保存時は画面の向きが横/再開時は画面の向きが縦 \\ \hline
    保存時はモバイル端末(iPhone12)/再開時はタブレット(iPad) \\ \hline
    保存時はタブレット(iPad)/再開時はモバイル端末(iPhone12) \\ \hline
    保存時はiOSのSafari/再開時はPCのChrome \\ \hline
    保存時はPCのChrome/再開時はiOSのSafari \\ \hline
    保存時と再開時の条件が同じ(Chrome拡張機能) \\ \hline
    保存時と再開時の条件が同じ(iOSアプリケーション) \\ \hline
    \end{tabular}
  \end{center}
\end{table}
% textlint-enable

% 実験に用いるWebページ
評価には複数の種類のWebサイトを用いる。
具体的には、一般的なWebページに加えて、1ページが長い(一番下までスクロールするために5秒以上かかる)Webページで検証する。
加えて、アニメーションが多用されているWebページ、SPA、その他の特殊なWebページをテストに使用する。
実験に用いるWebサイトを表\ref{tb:evl-basic-web-contents}に示す。

% textlint-disable
\begin{table}[htbp]
  \label{tb:evl-basic-web-contents}
  \caption{実験に用いるWebページ}
  \begin{center}
    \begin{tabular}{|l|l|}
    \hline
    種類 & WebページのURL  \\ \hline
    一般的なWebページ & \url{www.kgri.keio.ac.jp/research-synergies/talk-8.html} \\ \hline
    1ページが長いWebページ & \url{railstutorial.jp/chapters/beginning?version=5.1} \\ \hline
    アニメーションを多用しているWebページ & \url{ja.wikipedia.org/wiki/村井純} \\ \hline
    SPA & \url{plus-c.chuo-u.ac.jp/about/} \\ \hline
    特殊なWebサイト & \url{www.google.co.jp/maps/?hl=ja} \\ \hline
    \end{tabular}
  \end{center}
\end{table}
% textlint-enable
\subsection{動画の再生位置の復元}
本項では、本研究で定義した『動画の閲覧状態の復元』の評価方法について説明する。

まず、動画の再生位置の復元の評価に使用するWebページについて述べる。
動画の再生位置の復元の評価には、2021年のWebサイト別アクセス数ランキング\cite{The-50-Most-Visited-Websites-in-the-World}などの3つのサイト\cite{mmd-video-research}\cite{popular-video-service}を参考に、特に利用者が多く広く普及していると思われる動画サイト33個をピックアップし、使用する。
評価に用いる動画サービスの一覧を表\ref{tb:evl-video-service-list}に示す。
% なお、評価に使用した動画のURLは付録A\ref{chap:appendix-a}にすべて記載する。

% textlint-disable
\begin{table}[htbp]
  \label{tb:evl-video-service-list}
  \caption{評価に用いる動画サービス一覧}
  \begin{center}
    \begin{tabular}{|l|}
    \hline
    \multicolumn{1}{|c|}{\textbf{動画サービス}} \\\hline
    \begin{tabular}{l}
      Amazon Prime Video, Netflix, hulu, ABEMA, Paravi, TELASA, FODプレミアム, YouTube, XVideos, \\
      Pornhub, xnxx, Zoomレコーディング, xhamster, Twitch, bilibili, Tiktok, ニコニコ動画, FC2動画, \\
      Dailymotion, Gyao!, ツイキャス, パンドラTV, PeeVee, DMM, rakutenTV, TVer, Mirrativ, Mildom, \\
      OPENREC.tv, YOUKU, Sina Video, iQIYI, Tencent Video
    \end{tabular}\\\hline
  \end{tabular}
  \end{center}
\end{table}
% textlint-enable

次に、評価方法について説明する。
動画の再生位置の復元の評価では、実際にWebページ上の動画をブックマーク・再開し、動画の再生位置の復元の成功率を評価する。
『動画再生位置の復元』の定義は、第3章\ref{chap:web-snapshot-system-restore-definition}で説明した通りである。
なお、本実験の前提条件として、広告ブロッカーなどの拡張機能は無効の状態を仮定する。

動画の再生位置の復元の評価において、実験する条件を説明する。
本実験では、表\ref{tb:evl-video-audio-conditions}のように、Chrome拡張機能/iOSアプリケーションでの保存・復元を4パターン検証する。

% textlint-disable
\begin{table}[htbp]
  \label{tb:evl-video-audio-conditions}
  \caption{動画再生位置の復元の評価において実験する条件}
  \begin{center}
    \begin{tabular}{|l|}
    \hline
    \multicolumn{1}{|c|}{\textbf{実験条件}} \\\hline
    Chrome拡張機能で保存→Chrome拡張機能で復元 \\ \hline
    Chrome拡張機能で保存→iOSアプリケーションで復元 \\ \hline
    iOSアプリケーションで保存→Chrome拡張機能で復元 \\ \hline
    iOSアプリケーションで保存→iOSアプリケーションで復元 \\ \hline
    \end{tabular}
  \end{center}
\end{table}
% textlint-enable

\subsection{音声の再生位置の復元}
本項では、本研究で定義した『音声の閲覧状態の復元』機能について実験する。

\subsubsection{評価に使用するWebページ}
まず、音声の再生位置の復元の評価に使用するWebページについて述べる。
音声の再生位置の復元の評価には、『いちばんやさしい音声配信ビジネスの教本 人気講師が教える新しいメディアの基礎』\cite{}に掲載されている音声配信サービスのうち、Webブラウザ上で閲覧できる24個のサービスを使用した。
なお、評価に使用した音声配信サービスおよび音声のURLは付録A\ref{chap:appendix-a}にすべて記載する。

\subsubsection{評価方法}
次に、評価方法について説明する。
音声の再生位置の復元の評価では、実際にWebページ上の音声をブックマーク・再開し、音声の再生位置が復元することを確認する。
『音声再生位置の復元』の定義は、第3章\ref{chap:web-snapshot-system-restore-definition}で説明した通りである。

なお、本実験の前提条件として、広告ブロッカーなどの拡張機能は無効の状態を仮定する。

\subsubsection{実験する条件}
動画の再生位置の復元の評価において、実験する条件を説明する。
本実験では、表\ref{tb:evl-audio-audio-conditions}のように、Chrome拡張機能/iOSアプリケーションでの保存・復元を4パターン検証する。

% textlint-disable
\begin{table}[htbp]
  \label{tb:evl-audio-audio-conditions}
  \caption{音声再生位置の復元の評価において実験する条件}
  \begin{center}
    \begin{tabular}{|l|}
    \hline
    実験条件  \\ \hline
    Chrome拡張機能で保存→Chrome拡張機能で復元 \\ \hline
    Chrome拡張機能で保存→iOSアプリケーションで復元 \\ \hline
    iOSアプリケーションで保存→Chrome拡張機能で復元 \\ \hline
    iOSアプリケーションで保存→iOSアプリケーションで復元 \\ \hline
    \end{tabular}
  \end{center}
\end{table}
% textlint-enable

\subsection{PDFの復元}
PDFについては、再開時に保存時のページが開くことを検証する。

% 検証内容について(確率、ローカル、ch→ios, ch→ch)
PDFの保存機能はChrome拡張機能でのみサポートしている。
そのため、本実験ではChrome拡張機能で保存した後、Chrome拡張機能/iOSアプリケーションで再開し、それぞれ評価する。
本機能はOCRを用いて実現している。
PDFの中身やコンディションによってはOCRが失敗する可能性もある。
そのため、評価では1つのPDFにつき10回づつ実験し、成功率を示す。
また、本機能はWeb上にホスティングされているPDFとローカルPC上のPDFの両方に対応している。
ローカルPC上のPDFについては、同一PC内でのみ復元可能であるため、同一PC内での復元のみ検証する。
本評価でテストするパターンを表\ref{tb:evl-pdf-conditions}に示す。

本評価では、実際のユースケースを想定して様々な種類のPDFを実験に用いる。
具体的には、論文や書籍、漫画・説明書・契約書・IR資料・会社説明資料・スライドの復元を検証する。
検証に用いるPDFの一覧を表\ref{tb:evl-pdf-list}に示す。

% textlint-disable
\begin{table}[htbp]
  \label{tb:evl-pdf-conditions}
  \caption{実験する条件}
  \begin{center}
    \begin{tabular}{|l|}
    \hline
    実験条件  \\ \hline
    Chrome拡張機能で保存→Chromeで復元(ローカルPDF) \\ \hline
    Chrome拡張機能で保存→Chromeで復元 \\ \hline
    Chrome拡張機能で保存→iOSアプリケーションで復元 \\ \hline
    \end{tabular}
  \end{center}
\end{table}

\begin{table}[htbp]
  \label{tb:evl-pdf-list}
  \caption{実験に用いるPDF}
  \begin{center}
    \begin{tabular}{|l|l|}
    \hline
    種別 & PDFのリンク  \\ \hline
    論文 & 検証に使うPDFのURLを記入予定  \\ \hline
    書籍 & 検証に使うPDFのURLを記入予定  \\ \hline
    漫画 & 検証に使うPDFのURLを記入予定  \\ \hline
    説明書 & 検証に使うPDFのURLを記入予定  \\ \hline
    契約書 & 検証に使うPDFのURLを記入予定  \\ \hline
    IR資料 & 検証に使うPDFのURLを記入予定  \\ \hline
    会社説明資料 & 検証に使うPDFのURLを記入予定  \\ \hline
    スライド & 検証に使うPDFのURLを記入予定  \\ \hline
    \end{tabular}
  \end{center}
\end{table}
% textlint-enable

