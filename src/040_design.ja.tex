\chapter{設計}
\label{chap:design}
本章では、まずWeb-Snapshotシステム全体の構成について述べる。
ついで、クライアントであるiOSアプリケーション・Chrome拡張機能の設計について説明する。
そして、サーバ側のデータベース設計およびAPI設計を示す。

\section{全体の構成}
本研究で提案するWeb-Snapshotシステムは、ユーザが触れるインターフェイスであるクライアントと、サーバから構成される。

クライアントでは、iOSアプリケーションとChrome拡張機能に分かれている。
それぞれのクライアントアプリケーションは、ブックマーク保存モジュール・ブックマーク一覧モジュール・ブックマークの閲覧状態を復元するモジュールから成る。

サーバ側は、ユーザのブックマークデータを保存するためのデータベースサーバと、それらのデータを提供するアプリケーションサーバから成る。
アプリケーションサーバはクライアントが利用するためのWebAPIを提供する。
APIは以下の3種類のAPIを持つ。
\begin{enumerate}
   \item 認証関連API
   \item ブックマーク関連API
   \item フォルダ関連API
\end{enumerate}

図\ref{fig:design-system-overall}に本システムの全体像を示す。(TODO: 実装から抽象化した図に変える)
\begin{figure}[htbp]
  \caption{Web-Snapshotシステムの全体構成}
  \label{fig:design-system-overall}
  \begin{center}
    \includegraphics[bb=0 0 915 667,width=15cm]{img/design-system-overall.pdf}
  \end{center}
\end{figure}

\section{クライアント側設計}
本節では、クライアントを構成する3つのモジュールについて述べる。

\subsubsection{ブックマーク保存モジュール}
\subsubsection{ブックマーク一覧モジュール}
\subsubsection{ブックマーク閲覧状態復元モジュール}

\section{サーバ側設計}
本節では、サーバのデータベース設計およびAPI設計について説明する。

\subsection{データベース設計}

\subsubsection{ER図}

\subsection{API設計}
\subsubsection{認証API}
\subsubsection{コンテンツ関連API}
\subsubsection{フォルダ関連API}

\subsection{PDFからページ数を抽出する機能}
本項では、Webブラウザ上で閲覧中のPDFからページ数を抽出・保存する機能の設計について説明する。

Chrome等のブラウザが提供するPDFビューワーには、ページ数を取得するためのAPIが存在しない。そのため

そこで、本システムではユーザの画面のキャプチャを保存し、OCRを用いてそのキャプチャから閲覧中のPDFのページ数を取得する、という方法を提案する。

Webブラウザが提供するPDFビューワには、図\ref{fig:}のようにナビゲーションバーにに現在ユーザが閲覧しているページ数を表示する機能がある。
画面のキャプチャからこの数値を抽出することで、閲覧中のPDFのページ数を保存することができる。

OCRを用いて取得した文字列からPDFのページ数を抽出するためのアルゴリズムを図\ref{fig:}に示す。

\section{まとめ}
本章では、Web-Snapshotシステムの設計について述べた。次章では、本システムの実装について述べる。
